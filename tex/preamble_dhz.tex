\usepackage[left=3.0cm,right=3.0cm,top=2.5cm,bottom=2.5cm]{geometry}
    % \setcounter{tocdepth}{3}

\usepackage[ngerman,english]{babel} % ngerman: new German orthography use
                                    % english: the active language

\usepackage{datetime}               % DATETIME package must be loaded after BABEL.
    \renewcommand{\dateseparator}{.}
    \setdefaultdate{\ddmmyyyydate}

\usepackage[utf8]{inputenc}         % to input the special characters in most Latin languages
\usepackage[T1]{fontenc}

\usepackage[none]{hyphenat}         % for the usage of \nohyphens

\usepackage[nottoc]{tocbibind}      % to disable the inclusion of TOC title

\usepackage{graphicx}
\usepackage{float}
    \makeatletter\setlength{\@fptop}{0pt}\makeatother
\usepackage[format=hang]{caption}
\usepackage{subcaption}
\usepackage{diagbox}
\usepackage[export]{adjustbox}      % It is for 'max width=\textwidth' in \includegraphics.
\usepackage{fixltx2e}               % to fix the problem of math mode in moving arguments like in headings
\usepackage[bottom]{footmisc}       % to fix the problem that the footnote is not at the bottomm
                                    % when the paragraph that contains the footnote is followed
                                    % by figure who happens to appear on the next page.
\usepackage{tabu}                   % flexible LaTeX tabulars
\usepackage{multirow}
\usepackage{pdflscape}              % to produce landscape pages in a mainly portrait document


\usepackage{amsmath}
\usepackage{amssymb}
\usepackage{amsfonts}
\usepackage{dsfont}                 % double-stroke math font
\usepackage{mathrsfs}
\usepackage{amsthm}
% After the loading of "cleveref" comes the definition of new theorem environments
% in order to apply \cref to user-defined theorem environment.

% \usepackage{mdframed} % Be loaded after amsthm

\usepackage[usenames,dvipsnames,svgnames,table]{xcolor} % table option enables the cell coloring

\usepackage{xspace}

\usepackage{lastpage} % require HYPERREF package

\usepackage{varioref} % Be careful about the package order.
\usepackage{hyperref} % See section 12.1 of cleveref manual.
    \hypersetup{pdfstartview={XYZ null null 1.00}} % 100% zoom
\usepackage{cleveref}
    \crefname{equation}{equation}{equation}
    \Crefname{equation}{Equation}{Equation}
    \crefname{figure}{figure}{figure}
    \Crefname{figure}{Figure}{Figure}
    \crefname{table}{table}{table}
    \Crefname{table}{Table}{Table}
    % \crefname{chapter}{Chapter}{Chapter}
    % \Crefname{chapter}{Chapter}{Chapter}

    % [chapter] can be appended as an optional argument.
    \newtheorem{mythm}{Theorem}
    \newtheorem*{mythm*}{Theorem}
    \Crefname{mythm}{Theorem}{Theorem}
    \newtheorem{mydfn}{Definition}
    \newtheorem*{mydfn*}{Definition}
    \Crefname{mydfn}{Definition}{Definition}
    \newtheorem{myprt}{Property}
    \newtheorem*{myprt*}{Property}
    \Crefname{myprt}{Property}{Property}

\usepackage{algorithm}
\usepackage{algpseudocode}

\definecolor{mygrey}{gray}{0.6}
\usepackage{listings}
\lstset{
  language=matlab,
  basicstyle=\small,       % the size of the fonts that are used for the code
  keywordstyle=\sffamily,
  identifierstyle=,
  stringstyle=\color{mygrey},
  commentstyle=\small\slshape\sffamily\color{mygrey},
  %
  breakatwhitespace=false, % sets if automatic breaks should only happen at whitespace
  breaklines=true,         % sets automatic line breaking
  showspaces=false,        % show spaces everywhere adding particular underscores; it overrides 'showstringspaces'
  showstringspaces=false,  % underline spaces within strings only
  showtabs=false,          % show tabs within strings adding particular underscores
  %
  numbers=left,
  stepnumber=2,
  %
  % frame=single,            % adds a frame around the code  
  % frameround=tttt,
  literate=
  {Ö}{{\"O}}1
  {Ä}{{\"A}}1
  {Ü}{{\"U}}1
  {ß}{{\ss}}2
  {ü}{{\"u}}1
  {ä}{{\"a}}1
  {ö}{{\"o}}1
}


% operators for general usage
\DeclareMathOperator{\real}{Re}
\DeclareMathOperator{\imag}{Im}

% operators for Jacobian elliptic functions
\DeclareMathOperator{\sn}{sn}
\DeclareMathOperator{\cn}{cn}
\DeclareMathOperator{\dn}{dn}


% to tag the line in align environment
% usage: \numberthis\label{...}
\newcommand{\numberthis}{\addtocounter{equation}{1}\tag{\theequation}}

% roman "d" inside math mode
\newcommand{\mathrmd}[1][]{\mathrm{d}#1}

% differential operator in the integral inside math mode
\newcommand{\diff}[1][]{\,\mathrmd{#1}}

% imaginary unit
\newcommand{\im}[1][i]{\mathrm{#1}}

% transpose operator T
\newcommand{\tp}{\mathrm{T}}

\newcommand{\wrt}{with respect to\xspace} % with respect to
\newcommand{\rhs}{right-hand side\xspace} % right-hand side


% to color the content beneath the underbrace of \underbrace
% usage: in math mode \colorunderbrace[red]{aaaaaa}{bb}
\newcommand{\colorunderbrace}[3][black!60]{\textcolor{#1}{\underbrace{\textcolor{black}{#2}}_{#3}}}
\newcommand{\dateinfootnote}[1][\today]{\begingroup
                                \let\thefootnote\relax
                                \footnotetext{\sffamily\color{black!60} Date:~{\ifx&#1& \today \else{#1} \fi}}
                                \endgroup}
